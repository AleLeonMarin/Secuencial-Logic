\section{Sistemas Binarios}

\subsection{Sistemas Digitales}

Los sistemas digitales son aquellos que manipulan y trabajan con elementos discretos de información, 
representados de una forma binaria. Una característica importante de los sistemas digitales es su 
capacidad de manipular elementos discretos de información. Estos elementos se representan por 
cantidades físicas llamadas señales digitales. 

Casi todos los sistemas digitales están basados en dos valores discretos, los cuales conocemos como binarios. 
Un dígito binario puede tener dos valores: 0 o 1. Estos valores se les llaman bits. Estos bits son los que producen 
los códigos binarios que se utilizan en los sistemas digitales, a estos les llamamos códigos binarios.

\subsection*{HDL}

Es de suma importancia el conocimiento del lenguaje HDL (Hardware Description Language). Este lenguaje 
tiene como función la descripción de hardware. Se parece a un lenguaje de programación, 
permitiendo describir circuitos digitales en forma textual. Además, nos sirve para simular sistemas digitales 
y verificar su funcionamiento.

\begin{center}
    \begin{tikzpicture}[node distance=1.5cm]

        % Nodos sin figuras
        \node (main) {\textbf{Sistema Digital}};
        \node (sub1) [below of=main] {\textbf{Manejo de información discreta}};
        \node (sub2) [below of=sub1] {\textbf{Representación binaria}};

        % Flechas
        \draw[arrow] (main) -- (sub1);
        \draw[arrow] (sub1) -- (sub2);
        
    \end{tikzpicture}
\end{center}

\subsection{Números Binarios}