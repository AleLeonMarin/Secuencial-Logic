\section{Sistemas Binarios}

\subsection{Sistemas Digitales}

Los sistemas digitales son aquellos que manipulan y trabajan con elementos discretos de información, 
representados de una forma binaria. Una característica importante de los sistemas digitales es su 
capacidad de manipular elementos discretos de información. Estos elementos se representan por 
cantidades físicas llamadas señales digitales. 

Casi todos los sistemas digitales están basados en dos valores discretos, los cuales conocemos como binarios. 
Un dígito binario puede tener dos valores: 0 o 1. Estos valores se les llaman bits. Estos bits son los que producen 
los códigos binarios que se utilizan en los sistemas digitales, a estos les llamamos códigos binarios.

\subsection*{HDL}

Es de suma importancia el conocimiento del lenguaje HDL (Hardware Description Language). Este lenguaje 
tiene como función la descripción de hardware. Se parece a un lenguaje de programación, 
permitiendo describir circuitos digitales en forma textual. Además, nos sirve para simular sistemas digitales 
y verificar su funcionamiento.

\begin{center}
    \begin{tikzpicture}[node distance=1.5cm]

        % Nodos sin figuras
        \node (main) {\textbf{Sistema Digital}};
        \node (sub1) [below of=main] {\textbf{Manejo de información discreta}};
        \node (sub2) [below of=sub1] {\textbf{Representación binaria}};

        % Flechas
        \draw[arrow] (main) -- (sub1);
        \draw[arrow] (sub1) -- (sub2);
        
    \end{tikzpicture}
\end{center}

\subsection{Números Binarios}

En el sistema decimal conocemos que existen los millares, centenas, decenas, etc. Las cuales son 
potencias de 10 que están implícitas en la posición de los coeficientes. Si queremos ser más exactos, 
deberíamos escribir un número decimal así:

\begin{equation}
    1234 = 1 \cdot 10^3 + 2 \cdot 10^2 + 3 \cdot 10^1 + 4 \cdot 10^0
\end{equation}

No obstante, por la convención y facilidad de escritura, se escriben únicamente los coeficientes y se deducen 
las potencias necesarias de 10 de la posición que dichos coeficientes ocupan. Por lo tanto, un número decimal
con punto se representa con una serie de coeficientes, así:

\begin{equation}
    a_5a_4a_3a_2a_1a_0 \cdot a_{-1}a_{-2}a_{-3}a_{-4}a_{-5} 
\end{equation}

Los coeficientes $a_j$ son cualesquiera de los 10 dígitos decimales (0, ... , 9); el valor del subíndice j indica el 
valor de posición y, por lo tanto, la potencia 10 por la que se deberá multiplicar el coeficiente.

\begin{equation}
    10^5a_5 + 10^4a_4 + 10^3a_3 + 10^2a_2 + 10^1a_1 + 10^0a_0 + 10^{-1}a_{-1} + 10^{-2}a_{-2} + 10^{-3}a_{-3} 
\end{equation}

El sistema decimal actual es en base 10 porque usa 10 dígitos y los coeficientes se multiplican por 10. Pero el sistema
binario es en base 2 porque usa 2 dígitos, sus coeficientes solo pueden tener 2 valores: 0 o 1. Cada coeficiente se 
multiplicará por $2^j$ donde j es el valor del subíndice. Por lo tanto, un número binario se escribe así:

\begin{equation}
    a_n \cdot r^n + a_{n-1} \cdot r^{n-1} + ... + a_1 \cdot r^1 + a_0 \cdot r^0 + a_{-1} \cdot r^{-1} + a_{-2} \cdot r^{-2} + ... + a_{-m} \cdot r^{-m}
\end{equation}

Un ejemplo de un número binario es el siguiente:

\begin{equation}
    11010.11 = 1 \cdot 2^4 + 1 \cdot 2^3 + 0 \cdot 2^2 + 1 \cdot 2^0 + 0 \cdot 2^{-1} + 1 \cdot 2^{-2} = (26.75)_{10}
\end{equation}

El valor de los coeficientes $a_j$ varía entre 0 y $r-1$. Para poder distinguir entre números de diferentes bases, 
encerramos los coeficientes en paréntesis y se añade un subíndice que indica la base del número. 
Por ejemplo, el número $(4021.2)_5$

\begin{equation}
    (4021.2)_5 = 4 \cdot 5^3 + 0 \cdot 5^2 + 2 \cdot 5^1 + 1 \cdot 5^0 + 2 \cdot 5^{-1} = (511.4)_{10}
\end{equation}

\subsubsection{Bases Numéricas}

En el mundo actual usamos diferentes bases numéricas, las cuales nos facilitan la representación de números. 
Las bases que más se usan son las siguientes:

\begin{itemize}
    \item Bases 2 (Binario)
    \item Bases 8 (Octal)
    \item Bases 10 (Decimal)
    \item Bases 16 (Hexadecimal)
\end{itemize}

A continuación, se muestra una tabla con los valores de las bases numéricas:

\begin{table}[h]
    \centering
    \setlength{\tabcolsep}{12pt}
    \renewcommand{\arraystretch}{1.2}
    \begin{tabular}{|c|c|c|c|}
        \hline
        \textbf{Base Decimal} & \textbf{Base Binaria} & \textbf{Base Octal} & \textbf{Base Hexadecimal} \\
        \hline
        \texttt{00} & 0000 & 00 & 0 \\
        \texttt{01} & 0001 & 01 & 1 \\
        \texttt{02} & 0010 & 02 & 2 \\
        \texttt{03} & 0011 & 03 & 3 \\
        \texttt{04} & 0100 & 04 & 4 \\
        \texttt{05} & 0101 & 05 & 5 \\
        \texttt{06} & 0110 & 06 & 6 \\
        \texttt{07} & 0111 & 07 & 7 \\
        \texttt{08} & 1000 & 10 & 8 \\
        \texttt{09} & 1001 & 11 & 9 \\
        \texttt{10} & 1010 & 12 & A \\
        \texttt{11} & 1011 & 13 & B \\
        \texttt{12} & 1100 & 14 & C \\
        \texttt{13} & 1101 & 15 & D \\
        \texttt{14} & 1110 & 16 & E \\
        \texttt{15} & 1111 & 17 & F \\
        \hline
    \end{tabular}
\end{table}

\newpage

Aquí veremos unos ejemplos de conversión de bases numéricas:

\begin{itemize}
    \item \textbf{Binario a Decimal:} $(110101)_2 = 1 \cdot 2^5 + 1 \cdot 2^4 + 0 \cdot 2^3 + 1 \cdot 2^2 + 0 \cdot 2^1 + 1 \cdot 2^0 = (53)_{10}$
    \item \textbf{Octal a Decimal:} $(127.4)_8 = 1 \cdot 8^2 + 2 \cdot 8^1 + 7 \cdot 8^0 + 4 \cdot 8^{-1} = (87.5)_{10}$
    \item \textbf{Hexadecimal a Decimal:} $(B65F)_{16} = 11 \cdot 16^3 + 6 \cdot 16^2 + 5 \cdot 16^1 + 15 \cdot 16^0 = (46,687)_{10}$
    \item \textbf{Decimal a Binario:} $(13.75)_{10} = (1101.11)_2$
    \item \textbf{Decimal a Octal:} $(83.5)_{10} = (123.4)_8$
    \item \textbf{Decimal a Hexadecimal:} $(26.1875)_{10} = (1A.3)_{16}$
\end{itemize}

\subsubsection{Numeros Octales y Hexadecimales}


